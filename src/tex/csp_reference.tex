\documentclass[oneside]{book}
\usepackage{xcolor}
\definecolor{bg}{rgb}{0.95,0.95,0.95}
\definecolor{emphcolor}{rgb}{0.5,0.0,0.0}
\newcommand{\empha}{\bf\color{emphcolor}}
\usepackage{parskip}
\usepackage{minted}
\usepackage{caption}
\usepackage{amsmath}
\usepackage{amssymb}
\usepackage{amscd}
\usemintedstyle{friendly}
\setminted{bgcolor=bg,xleftmargin=15pt}
\usepackage{hyperref}
\hypersetup{pdftex,colorlinks=true,allcolors=blue}
\usepackage{hypcap}

\title{Communicating Sequential Processes\\ in C++ \\ Reference}
\author{John Skaller}
\begin{document}
\maketitle
\tableofcontents
\chapter{Global Rules}
The correct operation of the CSP system requires adherence to a set
of global rules and the programmer following, by discipline,
various protocols documented herein.

\subsection{No global data}
Use of global variables is prohibited.

\subsection{No system calls}
Calls to the operating system, whether directly or using
the C or C++ Standard libraries, are prohibited.

This means use of \verb%malloc%, file I/O, and system 
delays is prohibited. These services are provided instead
as part of the run time API. These operations can be used
with care is system extensions.

\subsection{Only transient use of machine stack}
The machine stack can only be used transiently.
The stack is defined as empty on entry to both \verb%resume% and
\verb%call% methods and must be empty when these method return
control. The \verb%resume% method of a continuation returns control
to perform a service call so any automatic variables on the machine
stack are lost. Conversely, during transient use of the machine
stack whilst evaluation expressions of blocks, service calls cannot
be performed without losing automatic variables.

The sole exception to this rule is that a new scheduler can be created
and run anywhere: the scheduler \verb%sync_run% method is a normal
C++ function calling which creates an independent nested csp system
which returns control only when all work is complete and the scheduler
terminates.

\chapter{Continuation Objects}
A CSP continuation is a pointer to an object which an instance 
of a class derived from the class %con_t%.

\begin{minted}{c++}
// continuation
struct con_t {
  con_t *caller; // caller continuation
  int pc;        // program counter
  union svc_req_t *svc_req; // request
  virtual con_t *resume()=0;
  virtual ~con_t(){}
};
\end{minted}

The derived class must have a method with the following signature:

\begin{minted}{c++}
con_t *call(con_t *caller_a, ...);
\end{minted}

where the ellipsis represents any other arguments the user wants.

The \verb%caller_a% parameter must hold the \verb%this% pointer of
the continuation to be invoked after this continuation is completed,
if any, or \verb%nullptr% if completion terminates the fibre owning
the continuation.

The call method must initialise the \verb%pc% non-static member
variable to 0, and set the \verb%caller% variable to the parameter
\verb%caller_a%. It should also assign a suitable initial value
to all user supplied non-static member variables of the derived class,
except the display variables which are initialised by the constructor.

Non-static member variables added to the class must have default
initialisers.

\section{Display}
A sequence of pointers may be passed to the \verb%constructor% routine after
which are pointers to the global object,
the most recent continuation object of the outermost ancestor, through
to the parent, in that order, and used to initialise non-static member
variables. This sequence is called the display.

The decision as to what the static nesting structure of the user 
program code is must be made by the user and must follow a consistent
pattern throughout the program in which containment forms a tree structure,
starting with the global object, and then a branch for each top level
procedure, then for each such branch its children.

Standard scoping rule them dictate that direct calls made from
a procedure may only be to its ancestors (including itself) or their
siblings.

Display pointer may not be modified.

Display pointers can be optimised away if they are not used.
The pointers may be pointers to const or non-const, however the pointers
themselves should be const to ensure the context of a continuation
is invariant (even if the content of the context is not). 

\section{Continuation State}
After construction, a continuation forms a closure over the context
specified by the display and is said to be in a {\em callable} state.
After the \verb%call% method is executed and prior to returning, it is said
to be in a {\em resumable} state. After returning, it is said
to be in a {\em dead} state.

A continuation can only be resumed if it is in a resumable state.
A continuation can be called if it is either dead or callable
and then becomes resumable. Care should be taken not to call
a dead continuation which is part of the closure context of
another continuation unless it is by design.


\section{Initialisation}
A CSP system is constructed by calling \verb%csp_run% with a resumable continuation
like this:
\begin{minted}{c++}
csp_run((new init)-> call(nullptr, &inlst, &outlst));
\end{minted}

\end{document}

