\documentclass[oneside]{book}
\usepackage{xcolor}
\definecolor{bg}{rgb}{0.95,0.95,0.95}
\definecolor{emphcolor}{rgb}{0.5,0.0,0.0}
\newcommand{\empha}{\bf\color{emphcolor}}
\usepackage{parskip}
\usepackage{minted}
\usepackage{caption}
\usepackage{amsmath}
\usepackage{amssymb}
\usepackage{amscd}
\usemintedstyle{friendly}
\setminted{bgcolor=bg,xleftmargin=15pt}
\usepackage{hyperref}
\hypersetup{pdftex,colorlinks=true,allcolors=blue}
\usepackage{hypcap}

\title{Communicating Sequential Processes\\ in C++}
\author{John Skaller}
\begin{document}
\maketitle
\tableofcontents
\chapter{CSP model}
Our CSP model uses a spaghetti stack of heap allocated continuation
objects linked together with pointers instead of a linear stack.
Each stack is owned by an object called a fibre.

Our usage protocols established in this document are primarily
focussed on memory management in a real time high performance
heavily concurrent system such as required for complex audio
signal processing.

When a subroutine is directly called, it is constructed on
the heap and initialised twice. The first phase of initialisation
establishes a binding to the most recent activation records
of its lexically scoped ancestors; the containing parent,
its parent, and so on up to and including a global context object.
The bindings are established by the contructor and are then invariant.

In the second phase, the arguments of the call are assigned to
local variables called parameters. Because we are using explicit
continuation passing, this will include the caller's current
continuation. The local program counter must be reset as well.

After initialisation, the resulting suspension is ready to run
and is executed by repeatedly calling the resume method.
This method returns a pointer to a continuation.

Inside the subroutine, the local program counter must be set before
returning the object this pointer. Then, the resume method will
execute a switch statement which jumps to the designated location
based on the value in the program counter. However the subroutine
may call another by constructing a new continuation object,
initialising it, and then returning a pointer to it. One of
the arguments to the call method must of course be the this
pointer. Returning control requires returning the saved caller
continuation pointer, however we must also require the continuation
object to delete itself.

\section{Ownership of closures}
In an FPL closures are managed by use of a garbage collector.
We must use a different strategy.

If a nested function closure is passed by the parent
to a down stack subroutine, a variable holding it is a
{\em weak pointer variable}. Note carefully weakness is property
of the variable not the pointer value.

\subsection{Passing closures as arguments}
A weak pointer variable is one whose lifetime is less than the
lifetime of the object to which its contained pointer points.
Since usually a nested function closure passed as an argument to
a subroutine has a longer lifetime than the called subroutine,
the parameter accepting the pointer is a weak variable.

A weak pointer variable is weaker than another if its own
lifetime is longer. Thus, it is safe to store a pointer obtained
from a weak pointer variable in a weaker one. In other words,
the called subroutine can pass its parameter to a subroutine it
calls. It is not necessarily safe to store the pointer in any
ancestor scope.

Weak pointer variables are standard C++ pointers. It is not
safe to delete the object contained in a weak pointer variable
nor is it necessary because the variable will become unreachable
before the object pointed at by its contents is deleted, or,
conversely, the object will always exist when the pointer is
accessed through the variable.

Rules of referential transparency also follow a related pattern.
The sole owner of an object is always free to modify it. However
once the owner shares access by storing the pointer in a weak
variable, it must not be modified to retain transparency.
When the share is lost due to the weak pointer variable being
destroyed, the owner may again modify the object.

Passing a pointer to a routine which follows weakness rules
is called lending the pointer by the provider, and borrowing
by the acceptor. The associated memory management rule is simply,
borrowed values can be lent, cannot be modified, and must not
be deleted by borrowers.

Note the immutability constraint can be lifted if that is the
intention but then referential transparency is lost.

\subsetion{Returning closures}
When a nested function is returned, it must accept ownership
of the function which returns it. The returning function
now must not delete itself on return as that would destroy
the context to which the nested function is bound.

Therefore, returning a closure also transfers responsibility
to propagate destruction: when the nested function closure is
destroyed it should also destroy its parent. Therefore a 
C++ \verb%unique_ptr% is an ideal vehicle for this purpose.

Unfortunately, this is where things get messy!
If a routine A has a child B which has a child C, and
A calls B which returns C, then C is bound to both B and
also A. The binding to B is an exclusive ownership, whereas
the binding to A is weak. If A now stores C in a local
variable, it cannot be a weak variable.


\end{document}

